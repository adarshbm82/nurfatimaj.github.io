
    \documentclass[a4, 11pt]{article}

    \usepackage{hyperref}
    \hypersetup{colorlinks=true, linkcolor=black, urlcolor=black}
    \usepackage[margin=1in]{geometry}
    \usepackage{float}

    \title{ {\Huge  Nurfatima Jandarova} }
    \date{}

    \begin{document}
    \maketitle

    % About and contact information

    \section*{Contact Information}
    \begin{table}[H]
      \begin{tabular}{ll}\href{https://www.eui.eu/}{European University Institute}&Phone: +39 331 566 3725\\
        via delle Fontanelle 18, San Domenico di Fiesole&
        Email: \href{mailto:Nurfatima.Jandarova [at] EUI.eu}{Nurfatima.Jandarova [at] EUI.eu} \\
        50014, Italy& Personal page: \href{https://nurfatimaj.com/}{https://nurfatimaj.com/} \\
        & Skype name: \href{skype:fdzahdarova?call}{fdzahdarova}
      \end{tabular}
    \end{table}

    \section*{Research interests}Economics of Education, Labour Economics, Gender Disparity

    % Education
    \section*{Education}\begin{table}[H]
        \begin{tabular}{p{2cm}p{14cm}}
            2016-2021 & \textbf{PhD in Economics}, \href{https://www.eui.eu}{European University Institute}\\ & Supervised by Prof. Andrea Ichino and Prof. Giacomo Calzolari\\
            2017 & \textbf{Genome-wide Data Analysis}, \href{https://www.tinbergen.nl/home}{Summer school at Tinbergen Institute}\\
            2014-2015 & \textbf{MSc in Economics}, \href{https://www.ucl.ac.uk}{University College London}\textit{ (with Distinction) }\\
            2009-2013 & \textbf{BA in Economics}, \href{https://www.kimep.kz/en}{KIMEP University}\textit{ (Summa Cum Laude) }\\
        \end{tabular}
      \end{table}

    % Experience
    \section*{Experience}
  \begin{table}[H]
      \begin{tabular}{p{2cm}p{14cm}}
          2018-2021 & Research Fellow at the \href{https://cla.umn.edu/economics}{Department of Economics, University of Minnesota} \\
          2017-Present & Research Assistant \href{https://www.eui.eu}{to Prof. Andrea Ichino, European University Institute} \\
          2017-2019 & Editorial Assistant \href{https://www.eui.eu}{to Prof. Andrea Ichino, European University Institute} \\
          2015-2016 & Chief Analyst at the \href{https://www.halykfinance.kz}{Research Department, JSC Halyk Finance} \\
          2012-2014 & Analyst at the \href{https://www.halykfinance.kz}{Research Department, JSC Halyk Finance} \\
      \end{tabular}
    \end{table}


    % Teaching
    \section*{Teaching experience}
  \begin{table}[H]
      \begin{tabular}{p{2cm}p{14cm}}
          2019 & {The Problem of Causality} at \href{https://www.eui.eu}{European University Institute}, Teaching Assistant to Prof. Andrea Ichino\\
          2018 & {Econometrics I} at \href{https://www.eui.eu}{European University Institute}, Teaching Assistant to Prof. Andrea Ichino\\
          2017 & {Econometrics I} at \href{https://www.eui.eu}{European University Institute}, Teaching Assistant to Prof. Andrea Ichino\\
      \end{tabular}
    \end{table}


    % Publications
    \section*{Research}\subsection*{Work in Progress}
      Jandarova. \textbf{Does intelligence shield from negative family shocks?}\\In this paper, I study the intergenerational effect of parental unemployment on children’s educational and labour market outcomes with an emphasis on heterogeneity of the effect by cognitive ability. The answer to this question is important to understand the characteristics of optimal policies. I find significant negative impact of parental job loss on all outcomes. I find that the effect on eductional outcomes is even more severe at the top of the ability distribution among children from disadvantaged background. Fruthermore, I find that education is not the only transmission channel and that labour market outcomes of more intelligent children are less affected by the parental job loss.\\
      Jandarova, \href{http://www.andreaichino.it}{Ichino}, \href{https://sites.google.com/site/aldorustichini/}{Rustichini}, Reuter, \href{https://sites.google.com/site/giuliozanella/}{Zanella}. \textbf{Genetic Endowment and Sorting into University Education}\\In the UK the number of students with a university degree has increased considerably between 1950 and 1970. Out of 100 high-school students, only about 3 went to university at the beginning of this period while about 8 did so in 1970 and about 19 in 1990. We know that, with few exceptions, only rich children had the possibility to attend college in 1950, independently of their skills. What we do not know is the answer to this question: did the expansion of enrollment after 1950 succeed in giving poor but smart children the opportunity to access a university? The objective of this research project is to answer this question. It is an important question because if, the answer is no, politicians need to think of better policies to achieve the goal of improving the opportunities of smart but poor children.\\
      Jandarova. \textbf{Unemployment and educational choices}\\This project studies the effect of labour market shocks on educational choices and whether these educational choices translate into more favourable outcomes later in life. For causal identification I exploit local employment shocks at the time of finishing compulsory schooling using the rich dataset on the population of university students in the UK from 1972 to 1993.\\


    % Seminars and conferences
    \section*{Seminars \& Conferences}
  \begin{table}[H]
      \begin{tabular}{p{2cm}p{14cm}}
          2017-2019 & {Microeconometrics Working Group} at {European University Institute}\\
      \end{tabular}
    \end{table}


    % Skills
    \section*{Skills}\begin{table}[H]
    \begin{tabular}{p{2cm}p{14cm}}
        Research Software & Stata, R, Python, Matlab, EViews, PLINK \\
        Computer Skills & LaTeX, git, bash, Moodle, HTML, Bootstrap, Hugo \\
        Language Skills & English, Russian, Kazakh, Italian \\
    \end{tabular}
  \end{table}


    % Awards and Grants
    \section*{Awards \& Grants}
  \begin{table}[H]
      \begin{tabular}{p{2cm}p{14cm}}
          2016-2020 & {Scholarship by the Italian Ministry of Foreign Affairs}{} to study at the PhD program at the European University Institute\\
          2014-2015 & {Bolashak International Scholarship of the President of the Republic of Kazakhstan}{} to study at the MSc program at the University College London\\
          2011-2012 & {KIMEP Presidential Scholarship}{} full tuition fee waiver\\
          2010-2011 & {KIMEP Presidential Scholarship}{} full tuition fee waiver\\
      \end{tabular}
    \end{table}

    \end{document}
  